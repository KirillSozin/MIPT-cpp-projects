\documentclass[a4paper,12pt]{article}
\usepackage[T2A]{fontenc}      
\usepackage{graphicx}
\usepackage{wrapfig}
\usepackage{hyperref}
\usepackage{tabularx}
\usepackage{amsfonts}
\usepackage{amsthm}
\usepackage[rgb]{xcolor}
\hypersetup{      
    colorlinks=true,    
  urlcolor=blue        
}
\usepackage[utf8]{inputenc}      
\usepackage[english,russian]{babel}  
\usepackage{amsmath,amsfonts,amssymb,amsthm,mathtools} 
\graphicspath{{pictures/}}
\DeclareGraphicsExtensions{.bmp,.png,.jpg}
\author{ Созинов Кирилл}
\title{ Время рассеяния атмосфер планет}
\date{\today}


\begin{document}

	\maketitle
	
	Как известно, концентрация молекул в силовом поле подчиняется закону Больцмана:
	
	\begin{equation}\label{Bol}
		n = n_0 e^{-\frac{\varepsilon_p}{k_BT}}
	\end{equation}
	
	
   	Применим закон распределения Больцмана к уединенной планете, окру­
женной газовой атмосферой. Последнюю будем считать изотермической. Кроме
того, будем предполагать, что все молекулы одинаковы. Это предположение не
лишает наши рассуждения общности, поскольку каждый газ (если его рассматри­
вать как идеальный), входящий в состав атмосферы, ведет себя независимо от
остальных газов. Будем считать, что масса атмосферы пренебрежимо мала по
сравнению с массой планеты. Тогда потенциальная энергия молекулы в поле
тяготения планеты будет равна -- $\frac{GMm}{r}$, Для концентрации молекул n на расстоянии r от центра планеты закон Больцмана (\ref{Bol}) дает:

\begin{equation}\label{XuiBol}
	n = n_0 e^{\frac{GMm}{k_BTr}}
\end{equation}

Однако, на бесконечности из формулы (\ref{XuiBol}) мы бы получили, что $n=n_0$. Это, очевидно, невозможно и формула была бы применима только для $n_0 = 0$, т.е. при отсутствии атмосферы

Данное противоречие связано с тем, что потенциальная энергия частицы в поле тяготения всегда остается конечной. Приняв ее за нуль, мы бы получили, что молекула совершает инфинитное движение (если ее полная энергия положительна). Такие молекулы просто не удерживаются полем тяготения планеты, при этом они всегда существуют. Так, к атмосфере планеты в целом неприменима формула Больцмана (\ref{Bol}), так как ее вывод предполагал, что газ находится в состоянии термодинамического равновесия.

Пусть в некоторый момент времени скорости всех частиц распределены по Максвеллу. Тогда "хвост" распределения -- частицы со скоростью выше второй космической (скорость ускользания) -- покинули бы планеты, а оставшиеся частицы были бы распределены по Больцману, в силу распределения скоростей по закону Максвелла. Однако, имеет смысл рассматривать лишь частицы с полной энергией $\varepsilon < \varepsilon_0 < 0$ -- тогда можно говорить о больцмановском распределении конечного числа частиц, устанавливающееся за конечное время.

Опишем вокруг планеты сферу $\sigma$ с настолько большим радиусом $r_0$, чтобы столкновениями снаружи можно было бы пренебречь. Тогда внутри сферы для всех молекул, за исключением молекул со скоростями выше скорости ускользания, справедливо распределение Максвелла - Больцмана. Рассмотрим скорости убегания на поверхности планеты ($v_o$) и на поверхности $\sigma$ ($v_{\sigma}$). Тогда:

\begin{equation}\label{EarthSpeed}
	v_0 = \sqrt{2g_0r_0}
\end{equation}

\begin{equation}\label{SigmaSpeed}
	v_{\sigma} = \sqrt{2g_{\sigma}r_{\sigma}} = r_0 \sqrt{\frac{2g_0}{r_{\sigma}}}
\end{equation}

Скорости $v_0$ и $v_{\sigma}$ связаны между собой уравнением энергии:

\begin{equation}\label{Energy}
	\frac{m{v_0}^2}{2} = \frac{m{v_{\sigma}}^2}{2} + \vartriangle\varepsilon_p
\end{equation}

Далее перейдем к безразмерной скорости $x = \frac{v}{v_m}$, где $v_m = \sqrt{\frac{2k_BT}{m}}$ - наиболее вероятная скорость. Тогда (\ref{Energy}) запишется как:

\begin{equation}\label{OtherEnergy}
	\frac{\vartriangle\varepsilon_p}{k_BT} = {x_0}^2 - {x_{\sigma}}^2
\end{equation}

Тогда, используя (\ref{OtherEnergy}) и (\ref{Bol}) получим:

\begin{equation}\label{concentration}
	n_{\sigma}e^{-{x_{\sigma}}^2} = n_0e^{-{x_0}^2},
\end{equation}

где $n_{\sigma}$ - концентрация молекул на сфере $\sigma$. Запишем также максвелловское распределение в безразмерных величинах:

\begin{equation}\label{Maksvell}
	dn = \frac{4}{\sqrt{\pi}}x^2e^{-x^2}dx
\end{equation}

Концентрация убегающих молекул на сфере $\sigma$ равна

\begin{equation}\label{SigmaConcentration}
	\vartriangle n = \frac{4n_{\sigma}}{\sqrt{\pi}}J
\end{equation}

где J это следующий интеграл:

\begin{equation}\label{integral}
	J = \int\limits_{x_{\sigma}}^{\infty} x^2e^{-x^2} dx
\end{equation}
 

Средняя безразмерная скорость таких молекул будет:

\begin{equation}\label{SrSpeed}
	c = {\langle x \rangle}_{x>x_{\sigma}} = \frac{1}{J}\int\limits_{x_{\sigma}}^{\infty} x^3e^{-x^2} dx = \frac{1}{2J}({x_{\sigma}}^2 + 1)e^{-{x_{\sigma}}^2}
\end{equation}

Найдем средний поток убегающих частиц Z, исходящий наружу от сферы о.
Поскольку распределение скоростей молекул изотропно, можно воспользоваться
формулой $z = \frac{n \langle v \rangle}{4}S$, или, $Z = \frac{1}{4} Scv_m\vartriangle n$. Подставив сюда выражения (\ref{SrSpeed}) и (\ref{SigmaConcentration}) и воспользовавшись (\ref{concentration}) получим:

\begin{equation}\label{potok}
	Z = 2\sqrt{\pi}(\frac{r_0}{r_{\sigma}}{x_0}^2 + 1)n_0{r_{\sigma}}^2v_me^{-{x_0}^2} = -\frac{dN}{dt}
\end{equation}

Концентрацию n можно выразить через N. Подавляющая масса атмосферы
приходится на тонкий слон, примыкающий к поверхности планеты. В пределах
этого слоя можно пренебречь кривизной поверхности планеты, а также изменением ускорения силы тяжести с высотой, т.е. положить $g = g_0$. Тогда из распределения Больцмана мы получаем

\begin{equation}
	N = 4\pi {r_0}^2n_0 \int\limits_0^{\infty} e^{-\frac{mg_0z}{k_BT}}dz = 4\pi {r_0}^2n_0 \frac{k_BT}{mg_0}
\end{equation}


Отсюда выражаем концентрацию $n_0$ и подставляем ее в выражение (\ref{potok}. Тогда, получаем равенство 

\begin{equation}\label{dNdt}
	\frac{dN}{dt} = \frac{N}{\tau}
\end{equation}   	

\begin{equation}\label{tau}
 \tau = \frac{2\sqrt{\pi} {r_0}^2 k_B T}{m g_0 {r_{\sigma}}^2v_m(\frac{r_0}{r_{\sigma}}{x_0}^2 + 1)}e^{{x_0}^2}
\end{equation}

 Проинтегрируем выражение (\ref{dNdt}) и получим:
 
 \begin{equation}\label{N}
 	N = N_0e^{-\frac{t}{\tau}}
 \end{equation}
 
 Из этой формулы видно, что постоянная $\tau$ имеет смысл времени, по истечении которого число молекул в атмосфере уменьшается в e раз. Поэтому $\tau$ может служить мерой времени, в течение которого планета может удерживать свою атмосферу.

Формула (\ref{tau}) еще не решает задачу, так как она содержит радиус $r_{\sigma}$, который мы еще не определили. В одном предельном случае решение очевидно. Это случай, когда планетная атмосфера — бесконечно разреженная. В ней пол­ностью отсутствуют столкновения между молекулами, а распределение молекул в пространстве и по скоростям устанавливается в результате столкновений с по­верхностью планеты. В таком случае $r_{\sigma} = r_0$. Тогда:

\begin{equation}\label{SimpleTau}
	\tau = \sqrt{2\pi r_0}{g_0} \frac{e^{{x_0}^2}}{x_0({x_0}^2 + 1)}
\end{equation}

В этом случае $\tau$ - время рассеяния бесконечно разреженной атмосферы.

Для того, чтобы определить радиус $r_{\sigma}$ в случае не разреженной атмосферы, введем какую-то длину l. Пусть она будет такова, что при r > l молекула становится не принадлежащей планете. Тогда понятно, что длина свободного пробега $\lambda = l$ при $r = r_{\sigma}$.

Чтобы определить l для реальной планеты, учтем, что она будет вращаться вокруг какой-то звезды (для блуждающих планет стоило бы рассмотреть ближайшее тело, которое своим притяжением может захватить молекулы). Пока сила притяжения планеты больше чем со стороны звезды (Солнца) $F_{\text{пл}} > F_c - F_{\text{ин}} $ молекула будет принадлежать планетее. Тогда возьмем l равный радиусу орбиты планеты и найдем $r_{\sigma}$ по формуле 

\begin{equation}
	r_{\sigma} = -\frac{RT}{GM\mu (\ln\lambda + A)} 
\end{equation}

\begin{equation}
	A = -(\ln\lambda_0 + \frac{GM\mu}{RTr_0})
\end{equation}

Тогда мы получим следующие высоты для газов из атмосферы:

\begin{table}[h!]

\begin{tabular}{|c|c|c|c|l}
\cline{1-4}
\textbf{Газ} & \textbf{Молекулярный вес} & \textbf{$r_{\sigma}$, км} & \textbf{Высота над земной поверхностью $h = r_{\sigma} - r_0$} &  \\ \cline{1-4}
$N_2$        & 28                        & 36700                     & 30300                                                          &  \\ \cline{1-4}
He           & 4                         & 10800                     & 4400                                                           &  \\ \cline{1-4}
$H_2$        & 2                         & 6700                      & 300                                                            &  \\ \cline{1-4}
\end{tabular}
\end{table}

Также построим зависимость высоты от молярной массы молекулы:

\begin{figure}[h!]
\includegraphics[width=17cm]{Graph1.jpg}
\end{figure}

График показывает, что если бы атмосфера состояла из одних только тя­
желых газов (тяжелее водорода), то сфера а практически совпадала бы с поверхностью Земли. В этом случае можно было бы пользоваться формулой (\ref{SimpleTau}) для бесконечно разреженной атмосферы. Для водорода радиус $r_{\sigma}$ значительно превышает радиус Земли $r_0$. В случае сложной атмосферы, состоящей из смеси различных газов, величина $r_{\sigma}$ определяется наиболее легким компонентом (с учетом, конечно, содержания этого компонента в атмосфере). Во всех случаях, представ­ляющих интерес, величина $x_0$ очень велика, и в формуле (\ref{tau}) можно пре­небречь единицей по сравнению с $\frac{r_0}{r_{\sigma}}{x_0}^2$. В этом приближении

\begin{equation}\label{TauRad}
	\frac{\tau_0}{\tau_{\sigma}} = \frac{r_{\sigma}}{r_0}
\end{equation}

Т.е. формула $\tau$ для бесконечно разреженной атмосферы завышает ответ только в несколько раз, и порядок величины вполне можно оценить с помощью формулы (\ref{SimpleTau}), чем мы и будем пользоваться далее.

Однако, эта же формула сильно зависит от температуры, по большей части из-за множителя $e^{{x_0}^2}$. Определить температуру, которую следует подставить, мы не сможем, т.к. она сильно колеблется от высоты и непостоянна во времени. Тогда решим обратную задачу: по данному $\tau$ найдем ${{x_0}^2}$ и выразим отсюда T. Сначала получаем следующее уравнение:

\begin{equation}\label{uravn}
	{x_0}^2 - \ln x_0 - \ln{({x_0}^2 + 1)} - \ln{\tau\sqrt{\frac{g_0}{2\pi r_0}}} = 0
\end{equation}

Его мы будем решать с помощью приближений программными средствами. Далее находим температуру по формуле:

\begin{equation}\label{temperature}
	T = \frac{g_0r_0\mu}{R{x_0}^2}
\end{equation}


Построим графики температур некоторых газов для некоторых небесных тел солнечной системы (см приложения 1-5)

Из них видно, что время $\tau$ очень чувствительно к изменениям температуры Т. При изменении Т на 12—15$\%$ $\tau$ меняется на два порядка. Отсюда следует, что рассеяние атмосфе­ры должно сильно возрастать из-за нерегулярных местных колебаний темпе­ратуры. Из графиков видно, что поле тяготения Земли надежно удерживает в течение геологи­ческих эпох все газы земной атмосферы, за исключением водорода и гелия.

Формула (\ref{temperature}) объясняет, почему Луна практически лишена атмосферы, а мощное гравитационное поле Юпитера не позволяет в течение геологиче­ских эпох рассеяться сколько-нибудь заметно даже наиболее легкому газу -- атомарному и молекулярному водороду. Понятно также, почему Луна лишена атмосферы, а на Титане — шестом спутнике Сатурна — обнаружена атмосфера из метана (СН,) и, возможно, аммиака (NH ), хотя скорости убегания на обоих спутниках почти одинаковы (2,4 км/с на Луне и 2,6 км/с на Титане).

Дело в том, что температура поверхности Титана (примерно 70—120 К) много ниже температуры лунной поверхности. При такой низкой темпера­туре только наиболее легкие газы -- водород и гелий -- обладают тепловыми
скоростями, достаточными для быстрого улетучивания их в окружающее
пространство.

  
\end{document}